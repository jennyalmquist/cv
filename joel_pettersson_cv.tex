%% Copyright 2007 Xavier Danaux (xdanaux@gmail.com).
%
% This work may be distributed and/or modified under the
% conditions of the LaTeX Project Public License version 1.3c,
% available at http://www.latex-project.org/lppl/.


\documentclass[12pt,a4paper]{moderncv}

% moderncv themes
\moderncvtheme[blue]{classic}                 % optional argument are 'blue' (default), 'orange', 'red', 'green', 'grey' and 'roman' (for roman fonts, instead of sans serif fonts)
%\moderncvtheme[green]{classic}                % idem

% character encoding
\usepackage[utf8]{inputenc}                   % replace by the encoding you are using

% adjust the page margins
\usepackage[scale=0.8]{geometry}
\recomputelengths                             % required when changes are made to page layout lengths

% personal data
\firstname{Joel}
\familyname{Pettersson}
\title{Curriculum Vitae}               % optional, remove the line if not wanted
\address{Redargatan 5 LÄG32}{120 61 Stockholm}    % optional, remove the line if not wanted
\mobile{070 - 339 79 89}                    % optional, remove the line if not wanted
%\phone{phone (optional)}                      % optional, remove the line if not wanted
%\fax{fax (optional)}                          % optional, remove the line if not wanted
\email{petterssonjoel@gmail.com}                      % optional, remove the line if not wanted
\extrainfo{\weblink{http://www.csc.kth.se/\textasciitilde joelpet}} % optional, remove the line if not wanted
%\photo[80pt]{profilbild}                         % '64pt' is the height the picture must be resized to and 'picture' is the name of the picture file; optional, remove the line if not wanted
%\quote{Some quote (optional)}                 % optional, remove the line if not wanted

%\nopagenumbers{}                             % uncomment to suppress automatic page numbering for CVs longer than one page


%----------------------------------------------------------------------------------
%            content
%----------------------------------------------------------------------------------
\begin{document}
\maketitle



\section{Utbildning}
\cventry{2007 -- idag}{Civilingenjörsutbildning i
datateknik}{KTH}{Stockholm}{}{Inlett tredje året på KTH:s datatekniklinje.}  % arguments 3 to 6 are optional
\cventry{2004 --2007}{Naturvetenskapligt gymnasium}{Jenny
Nyströmsskolan}{Kalmar}{}{Naturvetenskapsprogrammet (NVMD), inriktning Matematik
och datavetenskap.}  % arguments 3 to 6 are optional


\section{Erfarenheter}
\subsection{Serverhantering}
\cventry{2008 -- idag}{Experimentell Ubuntu-server}{}{}{}{Installation och
underhåll av privat experimentell Ubuntu-server med bl.a. LAMP.}
\subsection{Webbutveckling}
\cventry{2009}{www.JugglingFootballs.com}{Victor Rubilar}{}{}{Fotbollsjonglören
Victor Rubilar hade skapat en webbsidelayout i Photoshop som han behövde hjälp
med att överföra till webben samt att få funktioner på serversidan
implementerade.  Detta gjorde jag med hjälp av PHP-ramverket \emph{CakePHP} som
tillämpar designmönstret \emph{MVC}. Koden lät jag versionshanteras av Bazaar
VCS, och finns för närvarande att checka ut med kommandot \texttt{bzr branch
http://fb7.dyndns.org/victorrubilar/juggler}.
\weblink{http://www.jugglingfootballs.com}}
\cventry{2008 --2009}{Västerport Spa \& Relax:s webbplats}{Västerport
Relax}{Kalmar}{}{Det nystartade Västerport Relax i Kalmar önskade marknadsföra sig
via Internet och kontakade mig för att diskutera en webblösning. Ledd av
företagets önskemål på webbplatsens formgivning utgick jag ifrån en fritt
licensierad design och anpassade denna så att den mötte önskningarna.
Därefter hjälpte jag beställaren att fylla webbplatsen med dess innehåll.
\weblink{http://www.vasterportrelax.se}}


\section{Programmeringsspråk och dylikt}
\cvlanguage{Java}{Hemmastadd}{Programmerat i Java sedan gymnasiet och nu senast
på KTH.}
\cvlanguage{PHP}{Väl förtrogen}{Använt PHP i ett flertal webbprojekt sedan
gymnasiet.}
\cvlanguage{(X)HTML}{Väl förtrogen}{Följer ofta diskussioner kring åtkomlighet
och användarvänlighet.}
\cvlanguage{CSS}{Goda kunskaper}{Kan det mesta av CSS-standarden som stöds i
dagens webbläsare samt känner till några vanliga CSS-trick.}
\cvlanguage{Javascript}{Grundläggande}{Använt och modifierat en del
Javascript-kod, inklusive jQuery.}
\cvlanguage{SQL}{Grundläggande}{Hanterar de mest basala operationerna i
dagsläget och läser dessutom för närvarande en kurs i databasteknik.}


\section{Mjukvara}
\cvcomputer{Editor}{Emacs, \textbf{VIM}}{Utvecklinsmiljö}{\textbf{Eclipse}, Netbeans}
\cvcomputer{Versionshantering}{\textbf{Bazaar}, GIT,
SVN}{Operativsystem}{\textbf{Ubuntu}, Windows}



\section{Språk}
\cvlanguage{Svenska}{Modersmål}{Eventuellt med inslag av öländska}
\cvlanguage{Engelska}{Flytande}{Bland annat genom MVG i gymnasiets Engelska A-B}
\cvlanguage{Spanska}{Stapplande}{Kunskaperna från grundskolan är något rostiga}


\section{Intressen}
\cvline{Data \& IT}{\small Det har alltid funnits ett stort intresse för allt
vad datorer och informationsteknik innebär, vilket jag nu delvis får utlopp för
på KTH}
\cvline{Sport}{\small Min sportbana inleddes som fotbollsmålvakt på Öland, men
har nu övergått till volleyboll här i stan. Ett och annat styrketräningspass
försöker jag också hinna med, liksom all annan motion som kan tänkas dyka upp.}





%\section{Education}
%\cventry{year--year}{Degree}{Institution}{City}{\textit{Grade}}{Description}  % arguments 3 to 6 are optional
%\cventry{year--year}{Degree}{Institution}{City}{\textit{Grade}}{Description}  % arguments 3 to 6 are optional

%\section{Master thesis}
%\cvline{title}{\emph{Title}}
%\cvline{supervisors}{Supervisors}
%\cvline{description}{\small Short thesis abstract}

%\section{Experience}
%\subsection{Vocational}
%\cventry{year--year}{Job title}{Employer}{City}{}{Description}                % arguments 3 to 6 are optional
%\cventry{year--year}{Job title}{Employer}{City}{}{Description}                % arguments 3 to 6 are optional
%\subsection{Miscellaneous}
%\cventry{year--year}{Job title}{Employer}{City}{}{Description line 1\newline{}Description line 2}% arguments 3 to 6 are optional

%\section{Languages}
%\cvlanguage{language 1}{Skill level}{Comment}
%\cvlanguage{language 2}{Skill level}{Comment}
%\cvlanguage{language 3}{Skill level}{Comment}

%\section{Computer skills}
%\cvcomputer{category 1}{XXX, YYY, ZZZ}{category 4}{XXX, YYY, ZZZ}
%\cvcomputer{category 2}{XXX, YYY, ZZZ}{category 5}{XXX, YYY, ZZZ}
%\cvcomputer{category 3}{XXX, YYY, ZZZ}{category 6}{XXX, YYY, ZZZ}

%\section{Interests}
%\cvline{hobby 1}{\small Description}
%\cvline{hobby 2}{\small Description}
%\cvline{hobby 3}{\small Description}

%\closesection{}                   % needed to renewcommands
%\renewcommand{\listitemsymbol}{-} % change the symbol for lists

%\section{Extra 1}
%\cvlistitem{Item 1}
%\cvlistitem{Item 2}
%\cvlistitem[+]{Item 3}            % optional other symbol

%\section{Extra 2}
%\cvlistdoubleitem[\Neutral]{Item 1}{Item 4}
%\cvlistdoubleitem[\Neutral]{Item 2}{Item 5}
%\cvlistdoubleitem[\Neutral]{Item 3}{}

% Publications from a BibTeX file
%\nocite{*}
%\bibliographystyle{plain}
%\bibliography{publications}       % 'publications' is the name of a BibTeX file

\end{document}


%% end of file `template_en.tex'.
